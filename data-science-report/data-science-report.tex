\documentclass[a4paper]{article}

\usepackage[english]{babel}
\usepackage[utf8]{inputenc}
\usepackage{amsmath}
\usepackage{graphicx}
\usepackage[colorinlistoftodos]{todonotes}

\title{\#eMOVIX Data Science Report}

\author{Jordi Vilaplana}

\date{\today}

\begin{document}
\maketitle

\begin{abstract}
This document explains the different tasks to be achieved regarding the Twitter social network analysis part of the \#eMOVIX project.
\end{abstract}

\section{Define the questions}
\label{sec:questions_definition}



\section{Define the ideal data set}

The ideal data set would consist of the total amount of Catalan tweets produced in the world plus those tweets produced by the Twitter user accounts of interest.

\section{Determine what data you can access}

By using the Twitter Streaming API, up to 10\% of the total amount of tweets can be gathered.

Using the Twitter Search API, up to X tweets per day can be gathered.

\section{Obtain the data}

The data is obtained using the following applications:

emovix-twitter-streaming: https://github.com/eMOVIX/emovix-twitter-streaming

emovix-twitter-sarch: https://github.com/eMOVIX/emovix-twitter-search

\section{Clean the data}

An average tweet takes 4.26 KB of disk space when stored into the MongoDB database.

\section{Exploratory data analysis}

\section{Statistical prediction / modeling}

\section{Interpret results}

\section{Challenge results}

\section{Synthesize / write up results}

\section{Create reproducible code}

\section{Distribute results to other people}


\subsection{How to Leave Comments}

Comments can be added to the margins of the document using the \todo{Here's a comment in the margin!} todo command, as shown in the example on the right. You can also add inline comments:

\todo[inline, color=green!40]{This is an inline comment.}

\end{document}
